% ----------------------------------------------------------------------------
% Copyright (c) 2016 by Burkhardt Renz. All rights reserved.
% Die Vorlage für eine Abschlussarbeit in der Informatik am Fachbereich
% MNI der THM ist lizenziert unter einer Creative Commons
% Namensnennung-Nicht kommerziell 4.0 International Lizenz.
%
% Id:$
% ----------------------------------------------------------------------------

\chapter{Inoffizium}


\section{Einzug der Chargen}
\section{Begrüßung}

\enquote{Satis!}

\enquote{Silentium!} 

Liebe Damen, Gäste, Farben, Kartell- und Bundesbrüder,\\
als amtierender Senior des K.St.V Nassovia ist es mir ein innerliches 
Blumenpflücken, dieses Semester, am heutigen Abend, hier im Nassovenhaus 
gemeinsam mit der Giessener Burschenschaft Wartburg und meinen Conchargen
euch hier auf unserer Sommerkreuzkneipe dieses Semesters zu begrüßen.

Bevor wir weiter fortfahren wollen wir unsere Kehlen mit einem geziemten
Streifen aus unseren Gemäßen befeuchten.


\begin{quote}
Zum Wohl werte Corona!\\
Prost!
\end{quote}

Bei aller Freude und unserem fröhlichem Zusammensein,
weise ich euch darauf hin das eine Kneipe festen Regeln folgt\\
und daher bestimmte Verhaltensweisen nicht erwünscht sind.\\
Daher ergeht ein gestrenges:
\begin{quote}
    Non licet vagari,\\
    non licet fumare et\\
    non licet uti instrumenti radio agilis!\\
\end{quote}
\enquote{Das haben jetzt natürlich alle verstanden und deshalb können wir direkt weitermachen}

\begin{quote}
    Nein? Also gut:\\
    Aufstehen und Herumlaufen sowie Reden und Rauchen ist außerhalb 
    der Colloquien nicht gestattet. Handys sind auszuschalten.
\end{quote}

Um alle Unklarheiten zu bereinigen, sei hiermit klar gestellt,
dass das Kommando dieser Kneipe bei uns beiden Seniores und 
ausschließlich bei diesen liegt.\\
Der erste Schlag liegt bei uns beiden Seniores.
Der zweite Schlag liegt bei uns beiden Seniores und unseren Conchargen.
Der dritte Schlag liegt bei uns beiden Seniores und unseren Conchargen.

\begin{quote}
    Ad 1, ad 2, ad 3!\\
    Hochoffizium incipit!\\
    Omnes ad sedes!
\end{quote}

Werte Corona!\\
Das laufende Semester ist im vollem Gange. Ich hoffe, alle haben sich
gut in den Ferien erholt und sind bereits gut in das Semester gestartet.
\enquote{Sehr zum Wohl, werte Corona!}

Werte Corona, für die anstehenden Begrüßungen erschalle der Cantus 
\textbf{"Oh alte Burschenhherrlichkeit"}, zu finden auf Pagina 175
im KV-Liederbuch. Der Kantus beginnt mit seiner Ersten Strophe

\begin{quote}
    lieber Biersummikus, ich bitte diche, eine halbe Weise voraus...\\
    Satis!\\
    Ad primam! (Schlag)
    ...\\
    (Schlag) Cantus stad!\\
    Der Cantus steht bei seiner Ersten.\\
\end{quote}

Wir sitzen heute Abend alle nur hier, weil vor uns Generation von Studenten
sich entschlossen haben unserem Lebensbund
beizutreten und diesen, auch als \textbf{AHAH}, tatkräftig zu unterstützen.
Ich freue mich am heutigen Abend einen/einige Vertreter der Altherrenschaft
begrüßen zu dürfen.
\enquote{Liebe Alte Herren, auf euer Wohl!}
Der stehende Cantus zieht fort mit seiner Zweiten
\begin{quote}
    Ad scundam!(Schlag)\\
    ...
    (Schlag) Cantus stadt!\\
    Der Cantus steht bei seiner zweiten.
\end{quote}

\begin{quote}
    Ad tertiam!(Schlag!)\\
    ---\\
    (Schlag Cantus stat!)
\end{quote}

\begin{quote}
    Ad quartam!(Schlag)\\
    ...\\
    (Schlag) Cantus stat!
    Der Cantus steht bei seiner Vierten.
\end{quote}
\begin{quote}
    Ad quintam!ultimamque!(Schlag)
    ...\\
    (Schlag) Cantus ex!
    Cantus gebührend verklungen.
\end{quote}

Redeeinleitung

\section{Rede}
    Liebe Corona!
