\section{Grußworte}

Um die Stimmung so aufrecht zu erhalten und den nun folgenden
Grußworten einen angemessenen Rahmen zu bereiten, stimmen wir
den Cantus\\
\textbf{Gaudeamus Igitur}, zu finden auf Pagina 99 des KV Liederbuches
anzustimmen.
\begin{quote}
    \textbf{
        Lieber Biersummikus, eine halbe Weise voraus.\\
        Ad primam! (Schlag)\\
        ...\\
        (Schlag) Cantus stat!
        }
\end{quote}
An dieser Stelle möchte ich einen werten Bundesbruder aus dem
Altherrenvorstand um ein Grußwort bitten.\\
\dots\\
\textbf{Danke!}
\begin{quote}
    \textbf{
        Ad secundam! (Schlag)\\
        \dots\\
        (Schlag) Cantus stat!
        }
\end{quote}
(Nun bitte ich einen Vertreter einer verehrlichen Gießener Burschenschaft Wartburg un ein
 Grußwort.) \\
\textbf{Danke!}\\
(Ich bitte einen Vertreter einer verehrlichen freien gemischten Akademischen
Verbindung GeoGiessensis Gießen um ein Grußwort)\\
\dots\\
\textbf{Danke!}\\
(Auch einen Vertreter einer verehrlichen Gießener Burschenschaft Germania bitte
ich an dieser Stelle um ein Grußwort.)\\
\dots\\
\textbf{Danke}\\
\begin{quote}
    \textbf{
        Ad tertiam! (Schlag)\\
        \dots\\
        (Schlag) Cantus stad!
        }
\end{quote}
(Dem Vertreter eines verehrlichten W.K.St.V Cheruskia im UV Giessen
wird dies selbstverständlich nicht verwehrt)\\
\dots\\
\textbf{Danke}\\
\begin{quote}
    \textbf{
        Ad quartam! (Schlag)\\
        \dots\\
        (Schlag) Cantus stad!
        }
\end{quote}
Zudem bitte ich einen Vertreter eines verehrlichten K.St.V
Egbert im KV Trier um ein Grußwort.
\textbf{Danke!}
\begin{quote}
    \textbf{
        Ad quintam! (Schlag)\\
        Ad hexam! (Schlag)\\
        Ad heptam! ultimamque! (Schlag)
        }
\end{quote}