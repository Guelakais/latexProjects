\section{Begrüßung}

\begin{quote}
    \textbf{
        Satis!\\
        Silentium!
        }
\end{quote} 

Liebe Damen, Gäste, Farben, Kartell- und Bundesbrüder,\\
als amtierender Senior des K.St.V Nassovia ist es mir ein innerliches 
Blumenpflücken, dieses Semester, am heutigen Abend, hier im Nassovenhaus 
gemeinsam mit der Giessener Burschenschaft Wartburg und meinen Conchargen
euch hier auf unserer Sommerkreuzkneipe dieses Semesters zu begrüßen.

Bevor wir weiter fortfahren wollen wir unsere Kehlen mit einem geziemten
Streifen aus unseren Gemäßen befeuchten.


\begin{quote}
    \textbf{Zum Wohl werte Corona!}\\
    \textbf{Prost!}
\end{quote}

Bei aller Freude und unserem fröhlichem Zusammensein,
weise ich euch darauf hin das eine Kneipe festen Regeln folgt\\
und daher bestimmte Verhaltensweisen nicht erwünscht sind.\\
Daher ergeht ein gestrenges:
\begin{quote}
    \textbf{Non licet vagari,}\\
    \textbf{non licet fumare et}\\
    \textbf{non licet uti instrumenti radio agilis!}
\end{quote}
\enquote{Das haben jetzt natürlich alle verstanden.}
\begin{quote}
    Nein? Also gut:\\
    Aufstehen und Herumlaufen sowie Reden und Rauchen ist außerhalb 
    der Colloquien nicht gestattet. Handys sind auszuschalten.
\end{quote}

Um alle Unklarheiten zu bereinigen, sei hiermit klar gestellt,
dass das Kommando dieser Kneipe bei uns beiden Seniores und 
ausschließlich bei diesen liegt.\\
Der erste Schlag liegt bei uns beiden Seniores.
Der zweite Schlag liegt bei uns beiden Seniores.
Der dritte Schlag liegt bei uns beiden Seniores.

\begin{quote}
    \textbf{Ad 1, ad 2, ad 3!}\\
    \textbf{Hochoffizium incipit!}\\
    \textbf{Omnes ad sedes!}
\end{quote}

Werte Corona!\\ \\
Das laufende Semester ist im vollem Gange. Ich hoffe, alle haben sich
gut in den Ferien erholt und sind bereits ordentliche in das Semester 
gestartet.
\begin{quote}
    \textbf{Sehr zum Wohl, werte Corona!}
\end{quote}

Werte Corona, für die anstehenden Begrüßungen erschalle der Cantus \\
\textbf{Oh alte Burschenherrlichkeit}, zu finden auf Pagina 175
im KV-Liederbuch. Der Kantus beginnt mit seiner Ersten Strophe

\begin{quote}
    lieber Biersummikus, ich bitte diche, eine halbe Weise voraus...\\
    Satis!\\
    Ad primam! (Schlag)
    ...\\
    (Schlag) Cantus stad!\\
    Der Cantus steht bei seiner Ersten.\\
\end{quote}

Wir sitzen heute Abend alle nur hier, weil vor uns Generation von Studenten sich 
entschlossen haben unserem Lebensbund beizutreten und diesen, auch als 
\textbf{AHAH}, tatkräftig zu unterstützen. Ich freue mich am heutigen 
Abend einen/einige Vertreter der Altherrenschaft begrüßen zu dürfen.

\begin{quote}
    \textbf{Liebe Alte Herren, auf euer Wohl!}
\end{quote}
\begin{quote}
    Der stehende Cantus zieht fort mit seiner Zweiten.\\
    Ad scundam!(Schlag)\\
    ...
    (Schlag) Cantus stadt!\\
    Der Cantus steht bei seiner zweiten.
\end{quote}
(Es freut mich auch, dass heute Vertreter der verhlichen
Burschenschaft Germania Giessen anwesend sind.\\
\textbf{Sehr zum Wohl, Burschenschaft Germania})

(Es freut mich auch, heute Vertreter der verehrlichen
freien gmischten Akademischen Verbindung GeoGiessensis Giessen begrüßen zu 
dürfen \\
\textbf{Sehr zum Wohl, GeoGiessensis})

(Es freut mich auch, dass heute Vertreter der verehrlichen
Burschenschaft Adelphia Giessen anwesend sind.\\
\textbf{Sehr zum Wohl, Burschenschaft Adelphia})\\
\begin{quote}
    Der stehende Cantus zieht fort mit seiner dritten\\
    Ad tertiam!(Schlag!)\\
    ...\\
    (Schlag Cantus stat!) \\
    Der Cantus steht bei seiner dritten.\\
\end{quote}
Seit Jahren verbindet uns eine Freundschaft in Giessen zwischen unseren Vereinen,
die seinesgleichen sucht. Ich freue mich, dass ein verehrlicher W.K.St.V Unitas
Cheruskia Giessen im UV heute Abend anwesend ist.\\
\textbf{Sehr zum Wohl Cheruskia!}
\begin{quote}
    Der stehende Cantus zieht fort mit seiner Vierten\\
    Ad quartam!(Schlag)\\
    ...\\
    (Schlag) Cantus stat!\\
    Der Cantus steht bei seiner Vierten.
\end{quote}
(Desweiteren freue ich mich, einen Vertreter des verehrlichtent
K.St.V Egbert Trier begrüßen zu dürfen. Freddi schön, dass du da bist.\\
\textbf{Sehr zum Wohl, Egbert!})

Es ist schön, dass auch Damen anwesend sind. Ich möchte eauch euch herzlich Begrüßen.\\
\\
Ich freue mich außerdem meine BbBb aus der Aktivitas begrüßen zu dürfen. Wir sind eine
ordentliche Truppe und ich freue mich auf das weitere Semester mit euch.
\textbf{Auf euer Wohl}\\ \\
Besonders freue ich mich auch, dass noch viele andere Gäste am heutigen Abend 
den Weg auf unser Haus gefunden haben.\\
\textbf{Liebe Gäste, auf euer Wohl}\\
\begin{quote}
    Der stehende Cantus zieht fort mit seiner Fünften und bis zum Ende\\
    Ad quintam!(Schlag)\\
    ...\\
    Ad hexam!(Schlag)\\
    (Schlag) Cantus ex!\\
    Cantus gebührend verklungen.
\end{quote}
