Bevor ich euch in das Colloquium entlasse, ermahne ich euch Geistig,
wie auch körperlich auf die anstehende Rede ein zu stellen. So dann.

\begin{quote}
    Silentium ex! Colloquium! (Schlag)\\
    (Schlag) Coloquium ex! Silentium!
\end{quote}
\newpage
\section{Rede}
Alle wieder da? Gut. Bevor ich mit der Prinzipienrede fortfahre, lasst uns
den Cantus \textbf{Die Gedanken sind frei} Nassovenhaus zerschmetternd zum
Besten geben. Ihr findet ihn auf Pagina 264 im KV Liederbuch.
\begin{quote}
    \textbf{
        Biersummikus, bitte eine halbe Weise voraus.\\
        Ad primam! (Schlag)\\
        Ad secundam (Schlag)\\
        Ad tertiam (Schlag)\\
        Ad quartam (Schlag)\\
        Ad quintam ultimamque (Schlag)\\
        (Schlag) Cantus ex!\\
        Cantus feierlich verklungen.!
    }
\end{quote}
Dann können wir ja anfangen.
Sciencia wird das heute Abend besprochene Prinzip sein. Ich werde euch 
meine Eindrücke und Erfahrungen mit teilen und dieses Grundprizip 
zächst in einer theoretischen Richtung und dann die ganz praktische 
Umsetzung für uns Verbundungen herleiten.

In gewisser Weise ist Scientia der Konterpart zur Religio - 
oder wie beim UV als noch umfassenderes Prinzip des Virtus - 
weil es den Glauben und das akzeptieren von bestimmten gegebenheiten 
voraussetzt. Zbsp. Jesus ist der Sohn Gottes. Theismus wird voraus gesetzt.\\ \\
Scientia basiert auf dem Konzept der Agnostik. In der Regel werden vor allem
neue Erkenntnisse nicht einfach hin genommen und mehrfach nach geprüft.
So ist es im Studium wichtig vor allem in den Praktischen Teilen, genau das zu 
protokollieren, was man beobachtet hat und dies dann im Rahmen einer Fehlerdiskussion
zu relativieren, einzuordnen. Scientia lebt von der Überprüfbarkeit. Dabei ist stehen
sich Scientia und Religio nicht in gegenseitiger Ablehnung gegenüber. Whärend
Wissenschaft erstmal stumpf Fakten zusammenträgt hilft Religion bei der Sinnfrage.
Es hilft nichts, Bibelverse stumpf mit den Naturwissenschaften in Einklang bringen
zu wollen. Lieber sollte man den Glauben als eine Quelle der Kraft nutzen, wie
es auch zahlreiche hochrangige und gläubige Wissenschaftler getan haben. Genau diese
zeichnen sich durch einen ganz essenziellen Skill aus. Sie können gute Vorträge halten.

Genau hier kommen wir zur praktischen Umsetzung in den Studentenverbindungen.
Wir leben die Scientia durch Abhaltung wissenschaftlicher Vorträge in
den Semestern. Hier finden sich engagierte Bundesbrüder wie auch
Aussenstehende, die auf unseren Häusern über ein Thema ihrer Wahl, so
wissenschaftlich wie auch ansprechend wie möglich referieren um uns dieses
gut aufbereitet dar zu legen. Ich durfte in schon etlichen Vorträgen dieser 
Art lauschen. Da waren tolle Vorträge über Gerichtsbarkeit, die Bayrische 
Kultur oder Volkswirtschaftslehre dabei. Natürlich variert die Qualität dieser
mit ihren Referenden. Genau darum geht es. Wenn Bb ihre Vorträge hier
vermasseln, ist das nicht schlimm. Man zieht sich den Referenden Beiseite, dröselt
die Kritikpunkte auf und hilft ihm, damit es beim nächsten mal besser läuft.
Denn in der freien Wirtschaft oder auf einem Kongress ist man weit weniger gnädig.
Zudem geht es beisen Vorträgen auch steht um die Horizonterweiterung. Der Diskurs
wird gefördert und Teilnehmer solcher Vorträge lernen Themengebiete aus anderen
Fachbereichen kennen. Die Studiengänge sind heutzutage zum großen Teil spezialisiert.
So helfen auch diese Vorträge dabei, allen Beteiligten einen breiteren Blick auf die
Welt zu vermitteln.

Zum Schluss breitet sich hier die Vielfältigkeit dieses Prinzips aus. Natürlich ist
es kein Fremdkörper, der sich irgendwie unter unsere drei Grundprizipien gemogelt hat, 
sondern gehört genau da hin, wo es ist. Das praktizieren dessen liefert uns 
entscheidente Vorteile und ist am Ende wieder ein Grund dafür, warum es einen
positiven Effekt liefern kann, in einer Studentenverbindung zu sein. Damit lasse ich
euch jetzt zurück.
\begin{quote}
    \textbf{
        Silentium ex! Coloquium! (Schlag)\\
        (Schlag) Colloquium ex! Silentium
    }
\end{quote}