% ----------------------------------------------------------------------------
% Vorlage Abschlussarbeit Informatik THM (minimal)
%
% Copyright (c) 2016 by Burkhardt Renz. All rights reserved.
% Die Vorlage für eine Abschlussarbeit in der Informatik am Fachbereich
% MNI der THM ist lizenziert unter einer Creative Commons
% Namensnennung-Nicht kommerziell 4.0 International Lizenz.
%
% $Id: vorlage.tex 3835 2016-09-26 09:17:05Z br $
% ----------------------------------------------------------------------------

\documentclass[%
	BCOR=8.25mm,         % Bindekorrektur
	DIV=12,              % Satzspiegel
	parskip=half,				 % Abstand zwischen Absätzen
	bibliography=totoc,	 % Literaturverzeichnis im Inhaltsverzeichnis
	headsepline=on,      % Trennlinie Kolumnentitel
	]{scrbook}

%% Präambel
\usepackage[english, ngerman]{babel} % deutsche typogr. Regeln + Trenntabelle
\usepackage[T1]{fontenc}             % interner TeX-Font-Codierung
\usepackage{lmodern}                 % Font Latin Modern
\usepackage[utf8]{inputenc}          % Font-Codierung der Eingabedatei
\usepackage[babel]{csquotes}         % Anführungszeichen
\usepackage{graphicx}                % Graphiken
\usepackage{booktabs}                % Tabellen schöner
\usepackage{listingsutf8}            % Listings mit Einstellungen
\lstset{basicstyle=\small\ttfamily,
	tabsize=2,
	basewidth={0.5em,0.45em},
	extendedchars=true}
\usepackage{amsmath}	               % Mathematik
\usepackage[pdftex]{hyperref}       
\hypersetup{
	bookmarksopen=true,
	bookmarksopenlevel=3,
	colorlinks,
	citecolor=blue,
	linkcolor=blue,
}
\usepackage{scrhack}								 % unterdrückt Fehlermeldung von listings

%% Nummerierungstiefen
\setcounter{tocdepth}{3}             % 3 Stufen im Inhaltsverzeichnis
\setcounter{secnumdepth}{3} 		     % 3 Stufen in Abschnittnummerierung

% ----------------------------------------------------------------------------
\begin{document}

\frontmatter

%% Titelseite
\begin{titlepage}
	\begin{center}
	\includegraphics[width=0.9\textwidth]{img/mni-logo}\\[5cm]
	\textbf{\huge\sffamily CAP-Theorem}\\[2cm]
	\textsc{\Large Hausarbeit}\\Studiengang Bioinformatik\\[2cm]
	vorgelegt von\\
	\textbf{Steven H. G. Fleischer}\\ [1.5cm] 
	Mai 2022
	\end{center}
	\vfill
	\begin{tabular}{ll}
		Referent der Arbeit: & Prof. Dr. Donald E. Knuth\\ 
		Korreferent der Arbeit: & Prof. Dr. Leslie Lamport\\ 
	\end{tabular}
\end{titlepage}
\cleardoubleemptypage

%% Erklärung
\pagestyle{empty}
\begin{quote}
	\vspace*{4cm}

	\begin{center}
		\textbf{\Large\sffamily Eidesstattliche Erklärung}
	\end{center}

	Hiermit versichere ich, die vorliegende Arbeit selbstständig und unter
	ausschließlicher Verwendung der angegebenen Literatur und Hilfsmittel
	erstellt zu haben.

	Die Arbeit wurde bisher in gleicher oder ähnlicher Form keiner anderen
	Prüfungsbehörde vorgelegt und auch nicht veröffentlicht.

	\vspace{2em}

	Gießen, 12. September 2016
\end{quote}
\cleardoubleemptypage

%% Zusammenfassung
\pagestyle{empty}
\begin{quote}
	\vspace*{4cm}

	\begin{center}
		\textbf{\Large\sffamily Zusammenfassung}
	\end{center}

	Dieser Text beschreibt sich in einem gewissen Sinne selbst, nämlich
	wie die \LaTeX-Dateien aussehen, aus denen dieses Dokument erzeugt wird.

	Es geht also \emph{nicht} darum, wie man eine Abschlussarbeit
	gliedert, wie man in ihr argumentiert, wie man Konzepte illustriert
	usw.usf., sondern \emph{nur} darum, wie man das Manuskript der Arbeit
	in \LaTeX\ setzt. Deshalb ist anzuraten, dieses Dokument parallel mit
	seinen Quellen zu lesen, die in der Datei \verb=vorlage.zip= enthalten
	sind.

	Die \LaTeX-Datei basiert auf KOMA-Script von Markus Kohm. KOMA-Script
	verwendet europäische typografische Konventionen. In aller Regel
	werden in der Vorlage Standardeinstellungen von KOMA-Skript
	übernommen. Darüber hinaus wird versucht eine möglichst einfache
	Vorlage zu erstellen, die leicht an eigene Bedürfnisse angepasst
	werden kann --- ohne dass man tiefer gehende \LaTeX-Kenntnisse braucht.

	Die Verwendung stelle ich mir so vor: Für die eigene Abschlussarbeit
	kopiert man \verb=vorlage.tex= und passt die Datei entsprechend an.
	Für den eigentlichen Inhalt der eigenen Arbeit kann man die anderen
	\LaTeX-Dateien als Beispiele nehmen.
	
\end{quote}
\cleardoubleemptypage

%% Verzeichnissse
\tableofcontents
\listoffigures
\listoftables
\lstlistoflistings

\mainmatter 
\pagestyle{headings}

%\input{aufbau}
%\input{gliederung}
%\input{elemente}

\backmatter 

\appendix
%\input{install}
%\input{litverz}

\end{document}
% ----------------------------------------------------------------------------
