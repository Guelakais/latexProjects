\chapter{Logik}
\section{Ausagenlogische Formeln und Wharheitstafeln}
\subsection{Erfüllbarkeit}
Schreiben Sie für die folgenden zusammengesetzten Aussagen
(aussagenlogische Formeln) $\Phi$\textsubscript{1} bis $\Phi$\textsubscript{4}
die Wahrheitstafeln auf. Welche der Formeln sind erfüllbar? Gibt es es Tautologien
oder Kontradiktionen?
\begin{quote}
    \begin{enumerate}[label = {$\Phi$\textsubscript{\arabic*} =}]
        \item (A $\vee$ ($\neg$B))$\wedge$A\\
            {\tiny
            \begin{tabular}{cccc}
                AB & ($\neg$B) & A$\vee$($\neg$B) & (A $\vee$ ($\neg$B))$\wedge$A\\
                00 & 1 & 1 & 0 \\
                01 & 0 & 0 & 0 \\
                10 & 1 & 1 & 1 \\
                11 & 0 & 1 & 1 \\
            \end{tabular}
            Die Formal ist Erfüllbar, da sich in zwei Zeilen der Wahrheitstafel
            w ergibt, ist die Formel sogar für zwei unterschiedliche Belegungen
            der atomaren Variablen erfüllbar.}
        \item A $\vee$ ($\neg$(A$\wedge$B))\\
            {\tiny
            \begin{tabular}{ccc}
                AB & ($\neg$(A$\wedge$B)) & A $\vee$ ($\neg$(A$\wedge$B))\\
                00 & 1 & 1\\
                01 & 1 & 1\\
                10 & 1 & 1\\
                11 & 1 & 1\\
            \end{tabular}
            Erfüllbar und Tautologie, da die Formel für jede Belegung Wahr ist}
        \item (A$\vee$($\neg$B))$\wedge$($\neg$A)\\
            {\tiny
            \begin{tabular}{ccc}
                AB & (A$\vee$($\neg$B)) & (A$\vee$($\neg$B))$\wedge$($\neg$A)\\
                00 & 1 & 1\\
                01 & 0 & 0\\
                10 & 1 & 0\\
                11 & 1 & 0\\
            \end{tabular}
            Erfüllbar, da die Formel für eine Belegung der Atomaren Aussagen ein wahr
            ist.}
        \item (A$\wedge$B) $\wedge$(($\neg$A)$\vee$($\neg$B))\\
            {\tiny
            \begin{tabular}{ccc}
                AB & (A$\wedge$B) $\wedge$(($\neg$A)$\vee$($\neg$B))\\
                00 & 0 \\
                01 & 0 \\
                10 & 0 \\
                11 & 0 \\
            \end{tabular}
            Kontradiktion, da die Formel für keine der atomaren Aussagen wahr ist}
    \end{enumerate}
\end{quote}
\newpage
\subsection{logische Äquivalenzen}
Beweisen Sie mit Wahrheitstafeln die folgenden logischen Äquivalenzen
\begin{enumerate}
    \item (A$\leftrightarrow$B) = ((A$\to$B)$\wedge$(B$\to$A))\\
        {\tiny
        \begin{tabular}{ccc}
            AB & (A$\leftrightarrow$B) & ((A$\to$B)$\wedge$(B$\to$A))\\
            00 & 1 & 1 \\
            01 & 0 & 0 \\
            10 & 0 & 0 \\
            11 & 1 & 1 \\
        \end{tabular}
        }
    \item (A$\to$B) = ($\neg$A$\vee$B)\\
        {\tiny
        \begin{tabular}{cccc}
            AB & (A$\to$B) & ($\neg$A$\vee$B) & (A$\to$B) = ($\neg$A$\vee$B)\\
            00 & 1 & 1 & 1\\
            01 & 1 & 1 & 1\\
            10 & 0 & 0 & 0\\
            11 & 1 & 1 & 1\\
        \end{tabular}
        }
    \item (A$\vee$B) = ($\neg$($\neg$A$\wedge$$\neg$B))\\
    {\tiny
    \begin{tabular}{cccc}
        AB & (A$\vee$B) & ($\neg$($\neg$A$\wedge$$\neg$B))\\
        00 & 0 & 0\\
        01 & 1 & 1\\
        10 & 1 & 1\\
        11 & 1 & 1\\
    \end{tabular}
    }
\end{enumerate}
\subsection{Tautologien}
\begin{enumerate}
    \item A$\vee$($\neg$A) \\ Satz vom ausgeschlossenen Dritten\\
        {\tiny
        \begin{tabular}{cc}
            A & A$\vee$($\neg$A)\\
            0 & 1\\
            1 & 1\\
        \end{tabular}
        Tautologie, weil Formel für jede Aussage wahr ist}
    \item $\neg$(A$\wedge$($\neg$A)) \\ Satz vom Widerspruch\\
        {\tiny
        \begin{tabular}{cc}
            A & $\neg$(A$\wedge$($\neg$A))\\
            0 & 1 \\
            1 & 1 \\
        \end{tabular}
        Tautologie, weil Formel für jede Aussage wahr ist}
    \item ($\neg$($\neg$A)) $\leftrightarrow$ A \\ Satz von der doppelten Verneinung\\
        {\tiny
        \begin{tabular}{cc}
            A & ($\neg$($\neg$A)) $\leftrightarrow$ A\\
            0 & 1 \\
            1 & 1 \\
        \end{tabular}
        Tautologie, weil Formel für jede Aussage wahr ist}
    \item ($\neg$(A$\wedge$B)) $\leftrightarrow$ (($\neg$A)$\vee$($\neg$B)) \\
        {\tiny
        \begin{tabular}{cccc}
            AB & ($\neg$(A$\wedge$B)) & $\leftrightarrow$ (($\neg$A)$\vee$($\neg$B))
            & ($\neg$(A$\wedge$B)) $\leftrightarrow$ (($\neg$A)$\vee$($\neg$B))\\
            00 & 1 & 1 & 1 \\
            01 & 1 & 1 & 1 \\
            10 & 1 & 1 & 1 \\
            11 & 0 & 0 & 1 \\
        \end{tabular}
        \\
        Tautologie, da die Teilaussagen immer zueinander equivalente Aussagen liefern\\
        }
        ($\neg$(A$\vee$B)) $\leftrightarrow$ (($\neg$A)$\wedge$($\neg$B)) \\
        {\tiny
        \begin{tabular}{cccc}
            AB & ($\neg$(A$\vee$B)) & (($\neg$A)$\wedge$($\neg$B)) &
            ($\neg$(A$\vee$B)) $\leftrightarrow$ (($\neg$A)$\wedge$($\neg$B))\\
            00 & 1 & 1 & 1\\
            01 & 0 & 0 & 1\\
            10 & 0 & 0 & 1\\
            11 & 0 & 0 & 1\\
        \end{tabular}
        \\
        Tautologie, da die Teilaussagen immer zueinander equivalente Aussagen liefern}\\
        Sätze von De Morgan
    \item ((A$\to$B)$\wedge$A)$\to$B \\ Abtrennungsregel\\
    {\tiny
    \begin{tabular}{cccc}
        AB & (A$\to$B) & ((A$\to$B)$\wedge$A) & ((A$\to$B)$\wedge$A)$\to$B\\
        00 & 1 & 0 & 1\\
        01 & 0 & 0 & 0\\
        10 & 1 & 1 & 1\\
        11 & 1 & 1 & 1\\
    \end{tabular}
    \\Keine Tautologie, da die Formel nicht für jede Eingabe wahr ist}
    \item ((A$\to$B)$\wedge$($\neg$B))$\to$($\neg$A) \\ Widerlegungsregel\\
    {\tiny
    \begin{tabular}{ccccc}
        AB & ($\neg$B) & (A$\to$B) & ((A$\to$B)$\wedge$($\neg$B)) & 
        ((A$\to$B)$\wedge$($\neg$B))$\to$($\neg$A)\\
        00 & 1 & 1 & 1 & 1\\
        01 & 0 & 1 & 0 & 1\\
        10 & 1 & 0 & 0 & 1\\
        11 & 0 & 1 & 0 & 1\\
    \end{tabular}
    \\Ist eine Tautologie, da die Formel für jede Eingabe wahr ist}\\
    \item ((A$\to$B)$\wedge$(B$\to$C))$\to$(A$\to$B) \\ Kettenschlußregel\\
    {\tiny
    \begin{tabular}{ccccc}
        ABC & (A$\to$B) & (B$\to$C) & ((A$\to$B)$\wedge$(B$\to$C)) &
        ((A$\to$B)$\wedge$(B$\to$C))$\to$(A$\to$B)\\
        000 & 1 & 1 & 1 & 1\\
        001 & 1 & 1 & 1 & 1\\
        010 & 1 & 0 & 0 & 1\\
        011 & 1 & 1 & 1 & 1\\
        100 & 0 & 1 & 0 & 0\\
        101 & 0 & 1 & 0 & 1\\
        110 & 1 & 0 & 0 & 0\\
        111 & 1 & 1 & 1 & 1\\
    \end{tabular}
    \\ Erfüllbar und keine Tautologie, da die Formel nicht für jede Eingabe
    wahr ist}\\
\end{enumerate}
\subsection{technische Dokumente}
{
    \tiny
    \begin{tabular}{ccccc}
        LQBN & a & b & c & $\Phi$\\
        0000 & 0 & 0 & 0 & 0 \\
        0001 & 0 & 0 & 1 & 0 \\
        0010 & 0 & 1 & 0 & 0 \\
        0011 & 0 & 1 & 1 & 0 \\
        0100 & 1 & 0 & 0 & 0 \\
        0101 & 1 & 0 & 1 & 0 \\
        0110 & 1 & 1 & 0 & 0 \\
        0111 & 1 & 1 & 1 & 0 \\
        1000 & 0 & 0 & 0 & 0 \\
        1001 & 0 & 0 & 1 & 0 \\
        1010 & 0 & 1 & 0 & 0 \\
        1011 & 0 & 1 & 1 & 0 \\
        1100 & 1 & 0 & 0 & 0 \\
        1101 & 1 & 0 & 1 & 0 \\
        1110 & 1 & 1 & 0 & 0 \\
        1111 & 1 & 1 & 1 & 1 \\
    \end{tabular}
}
\section{Junktoren, Normalformen}
\subsection{logische Äquivalenzen mit Junktoren}
    Finden Sie eine Formel, die logisch äquivalent zu A $\oplus$ B ist und nur
    die Junktoren $\neg$, $\wedge$, $\vee$ enthält. (Hierbei ist $\oplus$ das
    exklusive Oder.) Hinweis: Arbeiten Sie mit einer Wahrheitstafel.
{
    \tiny
    \begin{tabular}{ccccc}
        AB & A $\oplus$ B & ($\neg$A)$\wedge$B & A$\wedge$($\neg$B) & 
        (($\neg$A)$\wedge$B)$\vee$(A$\wedge$($\neg$B))\\
        00 & 0 & 0 & 0 & 0\\
        01 & 1 & 1 & 0 & 1\\
        10 & 1 & 0 & 1 & 1\\
        11 & 0 & 0 & 0 & 0
    \end{tabular}
}
\subsection{Sheffer-Operator}
Stellen Sie zunächst den Junktor $\neg$ und anschließend den Junktor $\wedge$
mit dem Sheffer-Operator | NAND-Operator dar.\\
{
    \tiny
    \begin{tabular}{ccc}
        A & A|A & ($\neg$A)\\
        0 & 1 & 1\\
        1 & 0 & 0
    \end{tabular}
    \begin{tabular}{ccc}
        AB & (A|B)|(A|B) & A$\wedge$B\\
        00 & 0 & 0 \\
        01 & 0 & 0 \\
        10 & 0 & 0 \\
        11 & 1 & 1
    \end{tabular}
}
\subsection{DNF und KNF}
Stellen Sie zu der folgenden Wahrheitstafel eine aussagenlogische Formel $\phi$ in
disjunktiver und eine in konjunktiver Normalform auf.\\
{
    \tiny
    \begin{tabular}{cccccccc}
        xyz & 
        $\phi$ & 
        min\textsubscript{1}: ($\neg$a)$\wedge$y$\wedge$($\neg$z) & 
        min\textsubscript{2}: ($\neg$a)$\wedge$y$\wedge$z & 
        min\textsubscript{3}: x$\wedge$($\neg$y)$\wedge$($\neg$z) & 
        min\textsubscript{4}: x$\wedge$y$\wedge$($\neg$x) & 
        min\textsubscript{5}: x$\wedge$y$\wedge$z & 
        min\textsubscript{1}$\vee$min\textsubscript{2}$\vee$min\textsubscript{3}$\vee$min\textsubscript{4}$\vee$min\textsubscript{5}\\
        000 & 0 & 0 & 0 & 0 & 0 & 0 & 0\\
        001 & 0 & 0 & 0 & 0 & 0 & 0 & 0\\
        010 & 1 & 1 & 0 & 0 & 0 & 0 & 1\\
        011 & 1 & 0 & 1 & 0 & 0 & 0 & 1\\
        100 & 1 & 0 & 0 & 1 & 0 & 0 & 1\\
        101 & 0 & 0 & 0 & 0 & 0 & 0 & 0\\
        110 & 1 & 0 & 0 & 0 & 1 & 0 & 1\\
        111 & 1 & 0 & 0 & 0 & 0 & 1 & 1
    \end{tabular}
    \begin{tabular}{ccccccc}
        ->&
        max\textsubscript{1}: ($\neg$x)$\vee$y$\vee$($\neg$z)&
        max\textsubscript{2}: ($\neg$x)$\vee$y$\vee$z&
        max\textsubscript{3}: x$\vee$($\neg$y)$\vee$($\neg$z)&
        max\textsubscript{4}: x$\vee$y$\vee$($\neg$z)&
        max\textsubscript{5}: x$\vee$y$\vee$x&
        max\textsubscript{1}$\wedge$max\textsubscript{2}$\wedge$max\textsubscript{3}$\wedge$max\textsubscript{4}$\wedge$max\textsubscript{5}\\
        | & 1 & 1 & 1 & 1 & 0 & 0 \\
        | & 1 & 1 & 1 & 0 & 1 & 0 \\
        | & 1 & 1 & 1 & 1 & 1 & 1 \\
        | & 1 & 1 & 0 & 1 & 1 & 1 \\
        | & 1 & 0 & 1 & 1 & 1 & 1 \\
        | & 0 & 1 & 1 & 1 & 1 & 0 \\
        | & 1 & 1 & 1 & 1 & 1 & 1 \\
        | & 1 & 1 & 1 & 1 & 1 & 1 
    \end{tabular}
}
\section{Prädikate und Quantoren}
\subsection{Prädikatübersetzung}
es sei P(x) ein Prädikat und M = {a, b, c} die Grundmenge zu x. Zu den folgenden
Aussagen sollen logisch äquivalente aussagen angegeben werden, die keine Quantoren
enthalten.
a) 