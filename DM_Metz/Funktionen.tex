\chapter{Funktionen}
\section{Grundlagen des Funktionsbegriffs}
\subsection{injektiv,surjektiv und Bijektiv}
\begin{enumerate}
    \item f: $\mathbb{R}$ $\to$ $\mathbb{R}$ f(x) = x\textsuperscript{3}\\
    {
        \tiny
        injektiv, weil jeder Wert der Wertemenge mindestens einmal
        getroffen wird. Surjektiv, weil jeder Wert der Wertemenge
        höchstens einmal getroffen wird. Bijektiv, weil injektiv
        und Surjektiv
    }
    \item f: $\mathbb{Z}$ $\to$ $\mathbb{Z}$, f(x) = x\textsuperscript{3}\\
    {
        \tiny
        injektiv, weil jeder Wert der Wertemenge mindestens einmal
        getroffen wird. Surjektiv, weil jeder Wert der Wertemenge
        höchstens einmal getroffen wird. Bijektiv, weil injektiv
        und Surjektiv
    }
    \item f: $\mathbb{R}$ $\to$ $\mathbb{R}$, f(x) = x\textsuperscript{2}\\
    {
        \tiny
        nicht injektiv, weil nicht jeder Wert der Wertemenge mindestens
        einmal getroffen wird. Surjektiv, weil jeder Wertemenge höchstens
        einmal getroffen wird. Nicht Bijektiv, weil nicht injektiv und
        Surjektiv.
    }
    \item f: $\mathbb{R}$ $\to$ $\mathbb{R}$\textsuperscript{$\geq$0}, 
    f(x) = x\textsuperscript{2}\\
    {
        \tiny
        injektiv, weil jeder Wert der Wertemenge mindestens einmal getroffen
        wird. Surjektiv, weil jeder Wert der Wertemenge höchstens einmal
        getroffen wird. Bijektiv, weil injektiv und surjektiv.
    }
    \item f : $\mathbb{R}$\textsuperscript{$\geq$0} $\to$ $\mathbb{R}$, 
    f(x)= x\textsuperscript{2}\\
    {
        \tiny
        Surjektiv, weil jeder Wert der Wertemenge höchstens einmal getroffen
        wird. Nicht injektiv, weil nicht jeder Wert der Wertemenge getroffen
        wird. Nicht Bijektiv, weil nicht Injektiv und Surjektiv.
    }
    \item f : $\mathbb{R}$\textsuperscript{$\geq$0} $\to$ $\mathbb{R}$
    \textsuperscript{$\geq$0}, f(x) = x\textsuperscript{2}\\
    {
        \tiny
        Injektiv, weil jeder Wert der Wertemenge mindestens einmal getroffen
        wird. Surjektiv, weil jeder Wert der Wertemenge höchstens einmal
        getroffen wird. Bijektiv, weil injektiv und surjektiv.
    }
    \item f : $\mathbb{Z}$ $\to$ $\mathbb{N}\cup${0}, f(x) = x\textsuperscript{2}\\
    {
        \tiny
        Surjektiv, weil jeder Wert der Wertemenge höchstens einmal getroffen
        wird. Injektiv, wei jeder Wert der Wertemenge mindestens einmal
        getroffen wird. Bijektiv, weil injektiv und Surjektiv.
    }
    \item f : R\\
    {
        \tiny
        Surjektiv, weil jeder Wert der Wertemenge höchstens einmal getroffen
        wird. Injektiv, wei jeder Wert der Wertemenge mindestens einmal
        getroffen wird. Bijektiv, weil injektiv und Surjektiv.
    }
\end{enumerate}
\subsection{Abbildungen}
\begin{enumerate}
    \item f: $\mathbb{N}$\textsuperscript{$\leq$500} $\to$ $\mathbb{N}$, 
    f(x) = x\textsuperscript{2}\\
    \item f: $\mathbb{N}$ $\to$ $\mathbb{N}$, f(x) = x\textsuperscript{5}
    \item f: $\mathbb{N}$ $\to$ $\mathbb{N}$, f(x) = x\textsuperscript{3}
    \item f: $\mathbb{N}$ $\to$ $\mathbb{N}$, f(x) = x\textsuperscript{4}
\end{enumerate}
\subsection{M = {a,b,c,d} und $\triangle$\textsubscript{2} = {0,1}}
\begin{enumerate}
    \item M*M = {
        a{a,b,c,d},
        b{a,b,c,d},
        c{a,b,c,d},
        d{a,b,c,d},
    }\\ $\triangle$\textsubscript{2}*$\triangle$\textsubscript{2} ={
        00,01,11
    } M*$\triangle$\textsubscript{2} ={
        a0,a1,b0,b1,c0,c1,d0,d1,
    } $\triangle$\textsubscript{2}*M ={
        0a,0b,0c,0d,1a,1b,1c,1d,
    }
    \item Ja
        \item ja,
        \item ja,
        \item ja,
        \item ja,
\end{enumerate}
