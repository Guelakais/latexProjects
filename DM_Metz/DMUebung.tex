
\documentclass[%
	BCOR=8.25mm,         % Bindekorrektur
	DIV=12,              % Satzspiegel
	parskip=half,				 % Abstand zwischen Absätzen
	bibliography=totoc,	 % Literaturverzeichnis im Inhaltsverzeichnis
	headsepline=on,      % Trennlinie Kolumnentitel
	]{scrbook}

%% Präambel
\usepackage{enumitem}
\usepackage{tikz}
\usepackage{amsfonts}
\usepackage{amssymb}
\usepackage{fixltx2e}
\usepackage[english, ngerman]{babel} % deutsche typogr. Regeln + Trenntabelle
\usepackage[T1]{fontenc}             % interner TeX-Font-Codierung
\usepackage{lmodern}                 % Font Latin Modern
\usepackage[utf8]{inputenc}          % Font-Codierung der Eingabedatei
\usepackage[babel]{csquotes}         % Anführungszeichen
\usepackage{graphicx}                % Graphiken
\usepackage{booktabs}                % Tabellen schöner
\usepackage{listingsutf8}            % Listings mit Einstellungen
\lstset{basicstyle=\small\ttfamily,
	tabsize=2,
	basewidth={0.5em,0.45em},
	extendedchars=true}
\usepackage{amsmath}	               % Mathematik
\usepackage[pdftex]{hyperref}       
\hypersetup{
	bookmarksopen=true,
	bookmarksopenlevel=3,
	colorlinks,
	citecolor=blue,
	linkcolor=blue,
}
\usepackage{scrhack}								 % unterdrückt Fehlermeldung von listings

%% Nummerierungstiefen
\setcounter{tocdepth}{3}             % 3 Stufen im Inhaltsverzeichnis
\setcounter{secnumdepth}{3} 		     % 3 Stufen in Abschnittnummerierung

% ----------------------------------------------------------------------------
\begin{document}

\frontmatter

%% Titelseite
\begin{titlepage}
	\begin{center}
	\textsc{\Large Diskrete Mathematik Übungen}\\Studiengang Informatik\\[2cm]
	verfasst von\\
	\textbf{Steven H. G. Fleischer}\\ [1.5cm] 
	Mai 2022
	\end{center}
	\vfill
\end{titlepage}

%% Erklärung
\pagestyle{empty}
\begin{quote}
	\vspace*{4cm}

	\begin{center}
		\textbf{\Large\sffamily Eidesstattliche Erklärung}
	\end{center}

	Hiermit versichere ich, die vorliegende Arbeit selbstständig und unter
	ausschließlicher Verwendung der angegebenen Literatur und Hilfsmittel
	erstellt zu haben.
	Die Arbeit wurde bisher in gleicher oder ähnlicher Form keiner anderen
	Prüfungsbehörde vorgelegt und auch nicht veröffentlicht.

	\vspace{2em}

	Gießen, 07.05.2022
\end{quote}

%% Zusammenfassung


%% Verzeichnissse
\tableofcontents

\mainmatter 
\pagestyle{headings}
\chapter{Logik}
\section{Ausagenlogische Formeln und Wharheitstafeln}
\subsection{Erfüllbarkeit}
Schreiben Sie für die folgenden zusammengesetzten Aussagen
(aussagenlogische Formeln) $\Phi$\textsubscript{1} bis $\Phi$\textsubscript{4}
die Wahrheitstafeln auf. Welche der Formeln sind erfüllbar? Gibt es es Tautologien
oder Kontradiktionen?
\begin{quote}
    \begin{enumerate}[label = {$\Phi$\textsubscript{\arabic*} =}]
        \item (A $\vee$ ($\neg$B))$\wedge$A\\
            {\tiny
            \begin{tabular}{cccc}
                AB & ($\neg$B) & A$\vee$($\neg$B) & (A $\vee$ ($\neg$B))$\wedge$A\\
                00 & 1 & 1 & 0 \\
                01 & 0 & 0 & 0 \\
                10 & 1 & 1 & 1 \\
                11 & 0 & 1 & 1 \\
            \end{tabular}
            Die Formal ist Erfüllbar, da sich in zwei Zeilen der Wahrheitstafel
            w ergibt, ist die Formel sogar für zwei unterschiedliche Belegungen
            der atomaren Variablen erfüllbar.}
        \item A $\vee$ ($\neg$(A$\wedge$B))\\
            {\tiny
            \begin{tabular}{ccc}
                AB & ($\neg$(A$\wedge$B)) & A $\vee$ ($\neg$(A$\wedge$B))\\
                00 & 1 & 1\\
                01 & 1 & 1\\
                10 & 1 & 1\\
                11 & 1 & 1\\
            \end{tabular}
            Erfüllbar und Tautologie, da die Formel für jede Belegung Wahr ist}
        \item (A$\vee$($\neg$B))$\wedge$($\neg$A)\\
            {\tiny
            \begin{tabular}{ccc}
                AB & (A$\vee$($\neg$B)) & (A$\vee$($\neg$B))$\wedge$($\neg$A)\\
                00 & 1 & 1\\
                01 & 0 & 0\\
                10 & 1 & 0\\
                11 & 1 & 0\\
            \end{tabular}
            Erfüllbar, da die Formel für eine Belegung der Atomaren Aussagen ein wahr
            ist.}
        \item (A$\wedge$B) $\wedge$(($\neg$A)$\vee$($\neg$B))\\
            {\tiny
            \begin{tabular}{ccc}
                AB & (A$\wedge$B) $\wedge$(($\neg$A)$\vee$($\neg$B))\\
                00 & 0 \\
                01 & 0 \\
                10 & 0 \\
                11 & 0 \\
            \end{tabular}
            Kontradiktion, da die Formel für keine der atomaren Aussagen wahr ist}
    \end{enumerate}
\end{quote}
\newpage
\subsection{logische Äquivalenzen}
Beweisen Sie mit Wahrheitstafeln die folgenden logischen Äquivalenzen
\begin{enumerate}
    \item (A$\leftrightarrow$B) = ((A$\to$B)$\wedge$(B$\to$A))\\
        {\tiny
        \begin{tabular}{ccc}
            AB & (A$\leftrightarrow$B) & ((A$\to$B)$\wedge$(B$\to$A))\\
            00 & 1 & 1 \\
            01 & 0 & 0 \\
            10 & 0 & 0 \\
            11 & 1 & 1 \\
        \end{tabular}
        }
    \item (A$\to$B) = ($\neg$A$\vee$B)\\
        {\tiny
        \begin{tabular}{cccc}
            AB & (A$\to$B) & ($\neg$A$\vee$B) & (A$\to$B) = ($\neg$A$\vee$B)\\
            00 & 1 & 1 & 1\\
            01 & 1 & 1 & 1\\
            10 & 0 & 0 & 0\\
            11 & 1 & 1 & 1\\
        \end{tabular}
        }
    \item (A$\vee$B) = ($\neg$($\neg$A$\wedge$$\neg$B))\\
    {\tiny
    \begin{tabular}{cccc}
        AB & (A$\vee$B) & ($\neg$($\neg$A$\wedge$$\neg$B))\\
        00 & 0 & 0\\
        01 & 1 & 1\\
        10 & 1 & 1\\
        11 & 1 & 1\\
    \end{tabular}
    }
\end{enumerate}
\subsection{Tautologien}
\begin{enumerate}
    \item A$\vee$($\neg$A) \\ Satz vom ausgeschlossenen Dritten\\
        {\tiny
        \begin{tabular}{cc}
            A & A$\vee$($\neg$A)\\
            0 & 1\\
            1 & 1\\
        \end{tabular}
        Tautologie, weil Formel für jede Aussage wahr ist}
    \item $\neg$(A$\wedge$($\neg$A)) \\ Satz vom Widerspruch\\
        {\tiny
        \begin{tabular}{cc}
            A & $\neg$(A$\wedge$($\neg$A))\\
            0 & 1 \\
            1 & 1 \\
        \end{tabular}
        Tautologie, weil Formel für jede Aussage wahr ist}
    \item ($\neg$($\neg$A)) $\leftrightarrow$ A \\ Satz von der doppelten Verneinung\\
        {\tiny
        \begin{tabular}{cc}
            A & ($\neg$($\neg$A)) $\leftrightarrow$ A\\
            0 & 1 \\
            1 & 1 \\
        \end{tabular}
        Tautologie, weil Formel für jede Aussage wahr ist}
    \item ($\neg$(A$\wedge$B)) $\leftrightarrow$ (($\neg$A)$\vee$($\neg$B)) \\
        {\tiny
        \begin{tabular}{cccc}
            AB & ($\neg$(A$\wedge$B)) & $\leftrightarrow$ (($\neg$A)$\vee$($\neg$B))
            & ($\neg$(A$\wedge$B)) $\leftrightarrow$ (($\neg$A)$\vee$($\neg$B))\\
            00 & 1 & 1 & 1 \\
            01 & 1 & 1 & 1 \\
            10 & 1 & 1 & 1 \\
            11 & 0 & 0 & 1 \\
        \end{tabular}
        \\
        Tautologie, da die Teilaussagen immer zueinander equivalente Aussagen liefern\\
        }
        ($\neg$(A$\vee$B)) $\leftrightarrow$ (($\neg$A)$\wedge$($\neg$B)) \\
        {\tiny
        \begin{tabular}{cccc}
            AB & ($\neg$(A$\vee$B)) & (($\neg$A)$\wedge$($\neg$B)) &
            ($\neg$(A$\vee$B)) $\leftrightarrow$ (($\neg$A)$\wedge$($\neg$B))\\
            00 & 1 & 1 & 1\\
            01 & 0 & 0 & 1\\
            10 & 0 & 0 & 1\\
            11 & 0 & 0 & 1\\
        \end{tabular}
        \\
        Tautologie, da die Teilaussagen immer zueinander equivalente Aussagen liefern}\\
        Sätze von De Morgan
    \item ((A$\to$B)$\wedge$A)$\to$B \\ Abtrennungsregel\\
    {\tiny
    \begin{tabular}{cccc}
        AB & (A$\to$B) & ((A$\to$B)$\wedge$A) & ((A$\to$B)$\wedge$A)$\to$B\\
        00 & 1 & 0 & 1\\
        01 & 0 & 0 & 0\\
        10 & 1 & 1 & 1\\
        11 & 1 & 1 & 1\\
    \end{tabular}
    \\Keine Tautologie, da die Formel nicht für jede Eingabe wahr ist}
    \item ((A$\to$B)$\wedge$($\neg$B))$\to$($\neg$A) \\ Widerlegungsregel\\
    {\tiny
    \begin{tabular}{ccccc}
        AB & ($\neg$B) & (A$\to$B) & ((A$\to$B)$\wedge$($\neg$B)) & 
        ((A$\to$B)$\wedge$($\neg$B))$\to$($\neg$A)\\
        00 & 1 & 1 & 1 & 1\\
        01 & 0 & 1 & 0 & 1\\
        10 & 1 & 0 & 0 & 1\\
        11 & 0 & 1 & 0 & 1\\
    \end{tabular}
    \\Ist eine Tautologie, da die Formel für jede Eingabe wahr ist}\\
    \item ((A$\to$B)$\wedge$(B$\to$C))$\to$(A$\to$B) \\ Kettenschlußregel\\
    {\tiny
    \begin{tabular}{ccccc}
        ABC & (A$\to$B) & (B$\to$C) & ((A$\to$B)$\wedge$(B$\to$C)) &
        ((A$\to$B)$\wedge$(B$\to$C))$\to$(A$\to$B)\\
        000 & 1 & 1 & 1 & 1\\
        001 & 1 & 1 & 1 & 1\\
        010 & 1 & 0 & 0 & 1\\
        011 & 1 & 1 & 1 & 1\\
        100 & 0 & 1 & 0 & 0\\
        101 & 0 & 1 & 0 & 1\\
        110 & 1 & 0 & 0 & 0\\
        111 & 1 & 1 & 1 & 1\\
    \end{tabular}
    \\ Erfüllbar und keine Tautologie, da die Formel nicht für jede Eingabe
    wahr ist}\\
\end{enumerate}
\subsection{technische Dokumente}
{
    \tiny
    \begin{tabular}{ccccc}
        LQBN & a & b & c & $\Phi$\\
        0000 & 0 & 0 & 0 & 0 \\
        0001 & 0 & 0 & 1 & 0 \\
        0010 & 0 & 1 & 0 & 0 \\
        0011 & 0 & 1 & 1 & 0 \\
        0100 & 1 & 0 & 0 & 0 \\
        0101 & 1 & 0 & 1 & 0 \\
        0110 & 1 & 1 & 0 & 0 \\
        0111 & 1 & 1 & 1 & 0 \\
        1000 & 0 & 0 & 0 & 0 \\
        1001 & 0 & 0 & 1 & 0 \\
        1010 & 0 & 1 & 0 & 0 \\
        1011 & 0 & 1 & 1 & 0 \\
        1100 & 1 & 0 & 0 & 0 \\
        1101 & 1 & 0 & 1 & 0 \\
        1110 & 1 & 1 & 0 & 0 \\
        1111 & 1 & 1 & 1 & 1 \\
    \end{tabular}
}
\section{Junktoren, Normalformen}
\subsection{logische Äquivalenzen mit Junktoren}
    Finden Sie eine Formel, die logisch äquivalent zu A $\oplus$ B ist und nur
    die Junktoren $\neg$, $\wedge$, $\vee$ enthält. (Hierbei ist $\oplus$ das
    exklusive Oder.) Hinweis: Arbeiten Sie mit einer Wahrheitstafel.
{
    \tiny
    \begin{tabular}{ccccc}
        AB & A $\oplus$ B & ($\neg$A)$\wedge$B & A$\wedge$($\neg$B) & 
        (($\neg$A)$\wedge$B)$\vee$(A$\wedge$($\neg$B))\\
        00 & 0 & 0 & 0 & 0\\
        01 & 1 & 1 & 0 & 1\\
        10 & 1 & 0 & 1 & 1\\
        11 & 0 & 0 & 0 & 0
    \end{tabular}
}
\subsection{Sheffer-Operator}
Stellen Sie zunächst den Junktor $\neg$ und anschließend den Junktor $\wedge$
mit dem Sheffer-Operator | NAND-Operator dar.\\
{
    \tiny
    \begin{tabular}{ccc}
        A & A|A & ($\neg$A)\\
        0 & 1 & 1\\
        1 & 0 & 0
    \end{tabular}
    \begin{tabular}{ccc}
        AB & (A|B)|(A|B) & A$\wedge$B\\
        00 & 0 & 0 \\
        01 & 0 & 0 \\
        10 & 0 & 0 \\
        11 & 1 & 1
    \end{tabular}
}
\subsection{DNF und KNF}
Stellen Sie zu der folgenden Wahrheitstafel eine aussagenlogische Formel $\phi$ in
disjunktiver und eine in konjunktiver Normalform auf.\\
{
    \tiny
    \begin{tabular}{cccccccc}
        xyz & 
        $\phi$ & 
        min\textsubscript{1}: ($\neg$a)$\wedge$y$\wedge$($\neg$z) & 
        min\textsubscript{2}: ($\neg$a)$\wedge$y$\wedge$z & 
        min\textsubscript{3}: x$\wedge$($\neg$y)$\wedge$($\neg$z) & 
        min\textsubscript{4}: x$\wedge$y$\wedge$($\neg$x) & 
        min\textsubscript{5}: x$\wedge$y$\wedge$z & 
        min\textsubscript{1}$\vee$min\textsubscript{2}$\vee$min\textsubscript{3}$\vee$min\textsubscript{4}$\vee$min\textsubscript{5}\\
        000 & 0 & 0 & 0 & 0 & 0 & 0 & 0\\
        001 & 0 & 0 & 0 & 0 & 0 & 0 & 0\\
        010 & 1 & 1 & 0 & 0 & 0 & 0 & 1\\
        011 & 1 & 0 & 1 & 0 & 0 & 0 & 1\\
        100 & 1 & 0 & 0 & 1 & 0 & 0 & 1\\
        101 & 0 & 0 & 0 & 0 & 0 & 0 & 0\\
        110 & 1 & 0 & 0 & 0 & 1 & 0 & 1\\
        111 & 1 & 0 & 0 & 0 & 0 & 1 & 1
    \end{tabular}
    \begin{tabular}{ccccccc}
        ->&
        max\textsubscript{1}: ($\neg$x)$\vee$y$\vee$($\neg$z)&
        max\textsubscript{2}: ($\neg$x)$\vee$y$\vee$z&
        max\textsubscript{3}: x$\vee$($\neg$y)$\vee$($\neg$z)&
        max\textsubscript{4}: x$\vee$y$\vee$($\neg$z)&
        max\textsubscript{5}: x$\vee$y$\vee$x&
        max\textsubscript{1}$\wedge$max\textsubscript{2}$\wedge$max\textsubscript{3}$\wedge$max\textsubscript{4}$\wedge$max\textsubscript{5}\\
        | & 1 & 1 & 1 & 1 & 0 & 0 \\
        | & 1 & 1 & 1 & 0 & 1 & 0 \\
        | & 1 & 1 & 1 & 1 & 1 & 1 \\
        | & 1 & 1 & 0 & 1 & 1 & 1 \\
        | & 1 & 0 & 1 & 1 & 1 & 1 \\
        | & 0 & 1 & 1 & 1 & 1 & 0 \\
        | & 1 & 1 & 1 & 1 & 1 & 1 \\
        | & 1 & 1 & 1 & 1 & 1 & 1 
    \end{tabular}
}
\section{Prädikate und Quantoren}
\subsection{Prädikatübersetzung}
es sei P(x) ein Prädikat und M = {a, b, c} die Grundmenge zu x. Zu den folgenden
Aussagen sollen logisch äquivalente aussagen angegeben werden, die keine Quantoren
enthalten.
a) 
\chapter{Logik}

\section{Mengen}
\subsection{Teilmengen}
 
\begin{enumerate}[label = {\arabic* =}]
    \item A $\setminus$ B$\cup$C
    \item B $\setminus$ A$\cup$C
    \item C $\setminus$ A$\cup$B
    \item A$\cap$B $\setminus$ C
    \item B$\cap$C $\setminus$ A
    \item A$\cap$C $\setminus$ B
    \item 7 A$\cap$B$\cap$C
\end{enumerate}
\subsection{Mengenoperationen}
\begin{enumerate}
    \item {1,2,3}$\cup${1,3,5,7} = {1,2,3,5,7}
    \item {1,2,3,4}$\cap${1,3,5,7,9} = {1,3}
    \item {1,2,3,4,5} $\setminus${1,3,5,7} = {2,4}
    \item {1,2,4,8}$\cap${x,y,z}= 0
    \item ({1,2,3}$\cap${1,3,5})$\cup${x,y,z} = {1,3,x,y}
    \item ({1,2,3}$\cup${1,3,5})$\cap${x,y,z} = 0
    \item {1,5,10} $\setminus$ {x,y,z} = {1,5,10}
    \item {1,2,3,4,5}$\setminus$ ({1,2,3,4,5}$\cap${2,4,6}) = {1,3,5}
\end{enumerate}
\subsection{Mächtigkeit}
Es sei A = {-3,-2,-1,0,1,2,3} und B = {-4,-2,2,4}. Welche Mächtigkeiten
haben die Mengen:
\begin{itemize}
    \item |A| = 7
    \item |B| = 4
    \item |A$\cup$B| = |{-4,-3,-2,-1,0,1,2,3,4}|=9
    \item |A$\cap$B| = |{-2,2}| = 2
    \item |(A$\cap$B)$\cup$A| = |A|
    \item |A $\setminus$ B| = |{-3,-1,0,1,3}|=5
    \item |B $\setminus$ A| = |{-4,4}| =2
    \item |A$\cup$A$\cup$A| = |A|
    \item |AxA| = 49
    \item |BxB| = 16
    \item |AxB| = 28
    \item |BxA| = 28
\end{itemize}
\subsection{Mengenangabe}
    \begin{itemize}
        \item A = {1,2,3}
        \item B = {x,y}
        \item C = {0}
        \item M = AxB\textsuperscript{2}xC = {1,2,3}x{xx,yy,xy}x{0} =\\
        {
            (1,x,x,0),(1,x,y,0),(1,y,x,0),(1,y,y,0),\\
            (2,x,x,0),(2,x,y,0),(2,y,x,0),(2,y,y,0),\\
            (3,x,x,0),(3,x,y,0),(3,y,x,0),(3,y,y,0),
        }
    \end{itemize}
\subsection{Elemente von Mengen}

\chapter{Funktionen}
\section{Grundlagen des Funktionsbegriffs}
\subsection{injektiv,surjektiv und Bijektiv}
\begin{enumerate}
    \item f: R $\to$ R f(x) = x\textsuperscript{3}\\
    {
        \tiny
        injektiv, weil jeder Wert der Wertemenge mindestens einmal
        getroffen wird. Surjektiv, weil jeder Wert der Wertemenge
        höchstens einmal getroffen wird. Bijektiv, weil injektiv
        und Surjektiv
    }
    \item f: Z $\to$ Z, f(x) = x\textsuperscript{3}\\
    {
        \tiny
        injektiv, weil jeder Wert der Wertemenge mindestens einmal
        getroffen wird. Surjektiv, weil jeder Wert der Wertemenge
        höchstens einmal getroffen wird. Bijektiv, weil injektiv
        und Surjektiv
    }
    \item f: R $\to$ R, f(x) = x\textsuperscript{2}\\
    {
        \tiny
        nicht injektiv, weil nicht jeder Wert der Wertemenge mindestens
        einmal getroffen wird. Surjektiv, weil jeder Wertemenge höchstens
        einmal getroffen wird. Nicht Bijektiv, weil nicht injektiv und
        Surjektiv.
    }
    \item f: R $\to$ R\textsuperscript{$\geq$0}, f(x) = x\textsuperscript{2}\\
    {
        \tiny
        injektiv, weil jeder Wert der Wertemenge mindestens einmal getroffen
        wird. Surjektiv, weil jeder Wert der Wertemenge höchstens einmal
        getroffen wird. Bijektiv, weil injektiv und surjektiv.
    }
    \item f : R\textsuperscript{$\geq$0} $\to$ R, f(x)= x\textsuperscript{2}\\
    {
        \tiny
        Surjektiv, weil jeder Wert der Wertemenge höchstens einmal getroffen
        wird. Nicht injektiv, weil nicht jeder Wert der Wertemenge getroffen
        wird. Nicht Bijektiv, weil nicht Injektiv und Surjektiv.
    }
    \item f : R\textsuperscript{$\geq$0} $\to$ R\textsuperscript{$\geq$0}, 
    f(x) = x\textsuperscript{2}\\
    {
        \tiny
        Injektiv, weil jeder Wert der Wertemenge mindestens einmal getroffen
        wird. Surjektiv, weil jeder Wert der Wertemenge höchstens einmal
        getroffen wird. Bijektiv, weil injektiv und surjektiv.
    }
    \item f : Z $\to$ N$\cup${0}, f(x) = x\textsuperscript{2}\\
    {
        \tiny
        Surjektiv, weil jeder Wert der Wertemenge höchstens einmal getroffen
        wird. Injektiv, wei jeder Wert der Wertemenge mindestens einmal
        getroffen wird. Bijektiv, weil injektiv und Surjektiv.
    }
    \item f : R\\
    {
        \tiny
        Surjektiv, weil jeder Wert der Wertemenge höchstens einmal getroffen
        wird. Injektiv, wei jeder Wert der Wertemenge mindestens einmal
        getroffen wird. Bijektiv, weil injektiv und Surjektiv.
    }
\end{enumerate}
\subsection{Abbildungen}
\begin{enumerate}
    \item f: N\textsuperscript{$\leq$500} $\to$ N, f(x) = x\textsuperscript{2}\\
    \item f: N $\to$ N, f(x) = x\textsuperscript{5}
    \item f: N $\to$ N, f(x) = x\textsuperscript{3}
    \item f: N $\to$ N, f(x) = x\textsuperscript{4}
\end{enumerate}
\subsection{M = {a,b,c,d} und $\triangle$\textsubscript{2} = {0,1}}
\begin{enumerate}
    \item M*M = {
        a{a,b,c,d},
        b{a,b,c,d},
        c{a,b,c,d},
        d{a,b,c,d},
    }\\ $\triangle$\textsubscript{2}*$\triangle$\textsubscript{2} ={
        00,01,11
    } M*$\triangle$\textsubscript{2} ={
        a0,a1,b0,b1,c0,c1,d0,d1,
    } $\triangle$\textsubscript{2}*M ={
        0a,0b,0c,0d,1a,1b,1c,1d,
    }
    \item Ja
        \item ja,
        \item ja,
        \item ja,
        \item ja,
\end{enumerate}



\backmatter 


\end{document}
% ----------------------------------------------------------------------------
